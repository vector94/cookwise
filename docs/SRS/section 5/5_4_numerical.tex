\subsection{Prioritization Technique 3: Numerical Assignment}

\noindent\textbf{Description of Technique}

Numerical Assignment categorizes requirements into three priority groups: Critical (essential for system to function and deliver core value), Standard (important enhancements but not necessary for initial launch), and Optional (desirable features with limited impact if omitted).

\vspace{0.5cm}

\noindent\textbf{Application Process}

After completing the 100 Dollar Test and Ranking exercises, each team member independently categorized requirements from Section 4 into three priority groups based on technical foundation, core value proposition support, user experience enhancement, and deferability. The team then reached consensus through structured discussion.

\vspace{0.5cm}

\noindent\textbf{Results}

Table \ref{tab:numerical} presents the Numerical Assignment categorization of CookWise requirements.

\begin{table}[H]
\centering
\caption{Numerical Assignment Categorization for CookWise Requirements}
\label{tab:numerical}
\begin{tabular}{|l|l|c|}
\hline
\textbf{ID} & \textbf{Requirement Description} & \textbf{Category} \\
\hline
\multicolumn{3}{|c|}{\textbf{CRITICAL}} \\
\hline
PR1 & User authentication via external services & Critical \\
\hline
DL1 & Web scraping for product and pricing data & Critical \\
\hline
PR2 & AI-generated recipe suggestions based on sales & Critical \\
\hline
PR3 & Recipe filtering by cuisine and dietary restrictions & Critical \\
\hline
PR4 & Automatic shopping list generation from recipes & Critical \\
\hline
PR12 & Daily web scraping for product data & Critical \\
\hline
QR3 & Performance: Recipe generation response time & Critical \\
\hline
QR1 & Accuracy: Price data correctness & Critical \\
\hline
\multicolumn{3}{|c|}{\textbf{STANDARD}} \\
\hline
DL2 & GDPR compliance for user and partner data & Standard \\
\hline
DL3 & Single-store shopping constraint & Standard \\
\hline
PR7 & Display recipe details with instructions & Standard \\
\hline
PR8 & Store price comparisons and distances & Standard \\
\hline
PR9 & Display total savings and price differences & Standard \\
\hline
PR13 & Store user preferences for ingredients and recipes & Standard \\
\hline
QR4 & Usability: Time to complete first recipe selection & Standard \\
\hline
QR5 & Scalability: Concurrent user capacity & Standard \\
\hline
\multicolumn{3}{|c|}{\textbf{OPTIONAL}} \\
\hline
DL4 & Promotions and discounts in basket price & Optional \\
\hline
DL5 & Rate limits for external interfaces & Optional \\
\hline
DL6 & Domain events processing & Optional \\
\hline
PR5 & Record user ratings and feedback & Optional \\
\hline
PR6 & Store AI-generated recipes and details & Optional \\
\hline
PR10 & Display recipes by ingredient search & Optional \\
\hline
PR11 & Display additional ingredients with sale items first & Optional \\
\hline
PR14 & Interactive map with store locations and routes & Optional \\
\hline
PR15 & Display average recipe ratings & Optional \\
\hline
QR2 & Reliability: System uptime during peak hours & Optional \\
\hline
\end{tabular}
\end{table}

The Numerical Assignment categorization identified 8 Critical requirements (31\% of total), 8 Standard requirements (31\%), and 10 Optional requirements (38\%). This distribution shows a reasonable balance, with a focused set of Critical requirements deemed absolutely essential for the initial release, an equally sized group of Standard enhancements, and a larger set of Optional features that can be deferred to future releases.

Note that PR13 (Store user preferences) was classified as Standard rather than Critical, as while it enhances personalization significantly, the core functionality can operate without it for the initial MVP release.
