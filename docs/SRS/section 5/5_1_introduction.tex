\subsection{Introduction}

Requirements prioritization determines which features should be implemented first given constraints of time, budget, and resources. For CookWise, 41 requirements have been identified across four categories: 6 Domain Level Requirements (DL1-DL6), 14 Product Requirements (PR1-PR14), 13 Data Requirements (DR1-DR13), and 8 Quality Requirements (QR1-QR8). This section applies three prioritization techniques and compares results to establish a final priority ranking.

All three prioritization techniques were conducted internally by the CookWise development team (Project Manager, Customer Coordinator, and two project members) without involvement from external stakeholders. This internal prioritization approach allowed the team to efficiently establish initial priorities based on technical feasibility, business value, and strategic product goals.

The three techniques used are:
\begin{enumerate}
    \item \textbf{100 Dollar Test}: A ratio scale technique where stakeholders distribute 100 points across requirements based on their perceived value and importance.
    \item \textbf{Ranking}: An ordinal scale technique that assigns a unique rank from 1 to 41 to each requirement, with 1 being the highest priority.
    \item \textbf{Numerical Assignment}: An ordinal scale technique that groups requirements into three priority categories: Critical, Standard, and Optional.
\end{enumerate}
